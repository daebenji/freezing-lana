\documentclass[a4paper,10pt]{report}
%\documentclass[a4paper,10pt]{scrartcl}

\usepackage[utf8]{inputenc}
\usepackage{hyperref}

\title{Benjis Dokumentation}
\author{Benjamin Ludwig}
\date{}

\pdfinfo{%
  /Title    ()
  /Author   () 
  /Creator  ()
  /Producer ()	
  /Subject  ()
  /Keywords ()
}

\begin{document}
 
\maketitle
\hypersetup{
    colorlinks,
    citecolor=black,
    filecolor=black,
    linkcolor=black,
    urlcolor=black
    linktocpage
}
\tableofcontents
\chapter{Monitoring}
\section{Icinga Zeitprofile}


\begin{verbatim}

es wird eine Zeitperiode definiert, in der Alarmiert werden soll. 
Diese Periode ist dann mit 'check_period' auf den einzelnen Host 
oder Service anzuwenden. 
Im Beispiel soll immer alarmiert werden, AUSER von 05:00-06:25 
jeden Tag. 

Alarmierung für bestimmten Zeitpunkt abschalten:
\end{verbatim}

{\Large
\begin{verbatim}
define timeperiod {

     timeperiod_name 24x7_backup
     alias           immer-frueh
     sunday  00:00-05:00,06:25-24:00
     monday  00:00-05:00,06:25-24:00
     tuesday 00:00-05:00,06:25-24:00
     wednesday       00:00-05:00,06:25-24:00
     thursday        00:00-05:00,06:25-24:00
     friday  00:00-05:00,06:25-24:00
     saturday        00:00-05:00,06:25-24:00
}

\end{verbatim}
}
\chapter{sonstige Hacks}
\section{Unter Ubuntu jffs2-images mounten}
\begin{verbatim}
sudo apt-get install mtd-tools
sudo modprobe -v mtd
sudo modprobe -v jffs2
sudo modprobe -v mtdram total_size=256000 erase_size=256
sudo modprobe -v mtdchar
sudo modprobe -v mtdblock
sudo dd if=<deinImage.img> of=/dev/mtd0
sudo mount -t jffs2 /dev/mtdblock0 <deinPfadWoEsHinSoll>
\end{verbatim}

\section{Sed spielerei die Erste}
\begin{verbatim}
Achtung mit den Hochkommas!
Zeile an bestimmter Position einfügen(hier zeile 12) und dazu 
noch huebsch mit Tabulatoren formatieren:
sed ‘12i\\tTEXT\t\t\tMEHRTEXT’ <Datei>
\end{verbatim}
\section{Tunnel bauen}
\begin{verbatim}
#!/bin/bash
#build the tunnel to remote_ip via host
ssh -N -L <local_port>:<remote_ip>:<remote_port user@host &
#connect to host, via local port
ssh -p <local_port> <user>@localhost
#tunnel a remote port to another machine while using an existing tunnel
ssh -p <local_port> root@localhost -L localhost:8080:192.168.1.1:80

#scp durch bestehenden Tunnel
scp -P <local_port> <datei> root@localhost:<remote_pfad>
#oder vom remote host holen
scp -P <local_port> root@localhost:<remote_pfad> <lokaler_pfad>

\end{verbatim}

\section{expect-scripts}
\begin{verbatim}
#!/usr/bin/expect
\chapter{sonstige Hacks}
if {$argc != 1} {
    send_user "\tusage: $argv0 <ip-address>\n"
    exit
}

set IPADDRESS [lindex $argv 0]

# security: write password to root only readable file in e.g. /root/authfiles
# so you may use this password here by:
#
#set PASSWORD_DIR   /root/authfiles
#set PASSWORD_FILE  "pwd-${IPADDRESS}"
#set status [catch { exec cat ${PASSWORD_DIR}${PASSWORD_FILE} } PASSWORD]
#
# alternatively set password simply here
set PASSWORD "<password>"

spawn /usr/bin/ssh admin@${IPADDRESS}

while (1) {
    expect {
        "password:" {			
            send "${PASSWORD}\n"
            break
        }
        # this is useful, if ssh connects first time to IPADDRESS
        "connecting (yes/no)?" { send "yes\n" }
    }
}
expect "ES-2024PWR#" { send "show hardware-monitor c\n" }
expect "ES-2024PWR#" { send "exit\n" }

\end{verbatim}

\section{linux-default-settings}
\begin{verbatim}
um dein Linux etwas zu tunen, folgendes Skript ausführen.
# tell the kernel to only swap if it really needs it
sudo sysctl -w vm.swappiness="1"
# increase the number of allowed mmapped files
sudo sysctl -w vm.max_map_count="1048576"
# increase the number of file handles available globally
sudo sysctl -w fs.file-max="1048576"
# increase the number of sysv ipc slots for each type
sudo sysctl -w kernel.shmmax="65536"
sudo sysctl -w kernel.msgmax="65536"
sudo sysctl -w kernel.msgmnb="65536"
# allow up to 999999 processes with corresponding pids
# this looks nice and rarely rolls over - I've never had a problem with it.
sudo sysctl -w kernel.pid_max="999999" # unnecessary, but I like it
# seconds to delay after a kernel panic and before rebooting automatically
sudo sysctl -w kernel.panic="300"

# do not enable if your machines are not physically secured
# this can be used to force reboots, kill processses, cause kernel crashes, 
#etc without logging in
# but it's very handy when a machine is hung and you need to get control
# that said, I always enable it
kernel.sysrq="1"
sudo sysctl -w net.ipv4.ip_local_port_range="10000 65535"
sudo sysctl -w net.ipv4.tcp_window_scaling="1"
sudo sysctl -w net.ipv4.tcp_rmem="4096 87380 16777216"
sudo sysctl -w net.ipv4.tcp_wmem="4096 65536 16777216"
sudo sysctl -w net.core.rmem_max="16777216"
sudo sysctl -w net.core.wmem_max="16777216"
sudo sysctl -w net.core.netdev_max_backlog="2500"
sudo sysctl -w net.core.somaxconn="65000"

# these will need local tuning, currently set to start flushing dirty pages at 256MB
# writes will start blocking at 2GB of dirty data, but this should only ever happen if
# your disks are far slower than your software is writing data
# If you have an older kernel, you will need to s/bytes/ratio/ and adjust accordingly.
sudo sysctl -w vm.dirty_background_bytes="268435456"
sudo sysctl -w vm.dirty_bytes="1073741824"
\end{verbatim}

\section{rsync-magic}
\begin{verbatim}
logger -t Backup "begin incremental backup of <Directory>"
# incremental backup of /etc/apache2/*
rsync -chavz	P --stats /etc/apache2 \
<user>@<server>:<path_on_remote_host>
logger -t Backup "incremental backup done"
\end{verbatim}
\pagebreak

\section{Workspace Switcher Ubuntu 12.04 }
\begin{verbatim}
sudo apt-get install wmctrl
wmctrl -n 1
\end{verbatim}


\section{Mounten unter Linux}
\begin{verbatim}

place a credentials file at a place of your choise. in that case 
> /etc/backup-creds
put username and password in it as below.

cat /etc/backup-creds
username=<Domain>/<Password>
password=<password of $username>

Mount manually with:
mount -t cifs -o rw,nobrl,nosuid,nodev,credentials=</path_to_credentials file> \
<//backup-server/backup_path </local_mount_point/<local_backup_path/>

or put it in /etc/fstab for mounting it on bootstrap:
<//backup-server/backup_path  </local_mount_point/<local_backup_path/>  \
cifs    noauto,credentials=/etc/backup-creds    0       0
\end{verbatim}

\section{PDF Einschränkungen entfernen}
\begin{verbatim}

Entfert Drucksperren, editier und extrahier-einschränkungen auf PDFs.

1.  Install QPDF:
    > sudo aptitude install qpdf
2.  Remove restrictions:
    > qpdf --decrypt input.pdf output.pdf
3.  To do this with many PDFs use the following one-liner:
    > for file in *.pdf; do qpdf --decrypt $file ${file/.pdf/_rescued.pdf}; done

\end{verbatim}

\section{Ubuntu XFCE extended Screen}
\begin{verbatim}
1.  Install arandr:
    > sudo apt-get install arandr
2.  arandr von der comando-zeile aus starten. Ein GUI geht auf und dann die 
    Bildschirme zurecht rücken wie man es braucht.

Attention: on XFCE it's different. Go to Settings > Settings Editor
and select Displays. Browse the following > default > "your display" > 
Position > X (the value is where to begin to extend i.e. fullhd = 1920)
\end{verbatim}
\pagebreak
\section{Config Routing add/del}
\begin{verbatim}
 
Routen Setzen um Gateway im Entsprechenden Netz zu erreichen:
sudo route add -net 10.0.2.0/24 eth0
sudo route del -net 10.0.2.0 netmask 255.255.255.0 dev eth0

IP-Forwarding zwichen 2 Interfacen in Linux aktivieren
sysctl -w net.ipv4.ip_forward=1
echo 1 >/proc/sys/net/ipv4/ip_forward

\end{verbatim}



\section{ldap befehle zum abfragen}
\begin{verbatim}
 
Alle Benutzer listen:
ldapsearch -h host.domain(dc.foobar.com) -p 389 -x \
-b "ou=Mitarbeiter,ou=Benutzer,dc=domain,dc=com" \
-D "ldapbinduser@domain	" -w anonymous

Alle Gruppen und deren beinhaltende Benutzer listen:
ldapsearch -h host.domain(dc.foobar.com) -p 389 -x \
-b "ou=SicherheitsGruppen,ou=Benutzer,dc=domain(foobar),dc=com" 
\-D "ldapbinduser@domain" -w anonymous


\end{verbatim}
\section{esxi install e1000e Treiber für 82579LM}
\begin{verbatim}
 
Runterladen:
http://shell.peach.ne.jp/~aoyama/wordpress/download/net-e1000e-2.1.4.x86_64.vib

Datei per scp kopieren
scp *.vib root@esxi:/tmp

ESXi in maintenance mode schicken:
esxcli system maintenanceMode set -e true -t 0

Set the host acceptance level to CommunitySupported:
esxcli software acceptance set --level=CommunitySupported

Install the vib package:
esxcli software vib install -v /tmp/net-e1000e-2.1.4.x86_64.vib

Exit the ESXi from maintenance mode.
esxcli system maintenanceMode set -e false -t 0

Reboot

\end{verbatim}

\section{vmdk aus VMPlayer für ESXI konvertieren}
\begin{verbatim}

Basis-VM von VMPlayer zu ESXi konvertieren

1. Download des VMWare VDDK (Virtual Disk Development Kit) 
Download VDDK … Login erforderlich
2. Aufruf: vmware-vdiskmanager -r vmplayer.vmdk -t 4 esx(i).vmdk
3. Anm: Disksize verdoppelt sich ca.von 4GB auf 8GB

direkt auf dem ESXI Host geht es auch

1. Die vmdk in den Datastore kopieren
2. per SSH auf den ESXI Host verbinden
3. vmkfstools -i “/vmfs/volumes/Datastore/examplevm/examplevm.vmdk“ 
“/vmfs/volumes/Datastore 2/newexamplevm/newexamplevm.vmdk“

use -d thin if this was a thin provisioned client. you need to 
run this for every VMDK file if it’s thin provisioned in the directory. 
\end{document}
\end{verbatim}

\section{Swiss Macintosh Keyboard on Ubuntu 9.04 to 12.04 LTS (Precise Pangolin)}
\begin{verbatim}

Funktionsknopf:

1. Run the following command to append the configuration line to the file 
/etc/modprobe.d/hid_apple.conf creating it if necessary: 
  $ echo options hid_apple fnmode=2 | sudo tee -a /etc/modprobe.d/hid_apple.conf

2. Trigger copying the configuration into the initramfs bootfile.
  $ sudo update-initramfs -u -k all
  
grösser / kleiner als knopf:

* Create a new file ./.Xmodmap

vim ./.Xmodmap
keycode 49 = less greater less greater bar brokenbar bar 
! special section for Switzerland
keycode 91 = period period
keycode 94 = section degree

$ xmodmap ~/.Xmodmap
\end{verbatim}
\pagebreak

\section{Doku Wiki Authldap Plugin config}
\begin{verbatim}
server Adresse zum LDAP-Server. Entweder als Hostname (localhost)
oder als FQDN > ldap://dc.server.com:389

port Port des LDAP-Servers, falls kein Port angegeben wurde. 
trotzdem angeben wenn angegeben. > 389

usertree Zweig, in dem die Benutzeraccounts gespeichert sind.
> ou=Mitarbeiter,ou=Benutzer,dc=domain,dc=com    

grouptree Zweig, in dem die Benutzergruppen gespeichert sind. 
> ou=SicherheitsGruppen,ou=Benutzer,dc=domain,dc=de

userfilter LDAP-Filter, um die Benutzeraccounts zu suchen.
> (userPrincipalName=%{user}@domain.com)

groupfilter LDAP-Filter, um die Benutzergruppen zu suchen. 
> (&(cn=*)(Member=%{dn})(objectClass=group))

version Zu verwendende Protokollversion von LDAP. > 3   

starttls Verbindung über TLS aufbauen? > nicht ankreutzen 

referrals Weiterverfolgen von LDAP-Referrals (Verweise)? > nicht ankreutzen

binddn DN eines optionalen Benutzers, wenn der anonyme Zugriff 
nicht ausreichend ist. > ldapbinduser@domain.com

bindpw Passwort des angegebenen Benutzers. > passwort

userscope Die Suchweite nach Benutzeraccounts. > sub

groupscope Die Suchweite nach Benutzergruppen. > sub

groupkey Gruppieren der Benutzeraccounts anhand eines beliebigen 
Benutzerattributes z. B. > cn

\end{verbatim}


\chapter{Datenbanken}

\section{Postgres DB - HBA config}
\begin{verbatim}

für Postgresql gibt es eine Datei /etc/postgresql/<VERSION>/main/pg_hba.conf
die als Art "Firewall" Funktion für die Datenbank funktioniert.

Standardmässig besagt diese das Verbindungen ausschliesslich von Lokal auf die
Datenbank gemacht werden dürfen.

um dies zu Ändern muss die entsprechende IP oder das Netz angegeben werden:

# Database administrative login by UNIX sockets
local   all         postgres                         trust

# TYPE  DATABASE    USER        CIDR-ADDRESS          METHOD

# "local" is for Unix domain socket connections only
#local  all         all                               ident
local   all         all                               ident
# IPv4 local connections:
#host   all         all         127.0.0.1/32          md5
host    all         all         127.0.0.1/32          md5
host    all         all         192.168.0.1/24        trust
# IPv6 local connections:
#host   all         all         ::1/128               md5
host    all         all         ::1/128               md5
host    all         all         192.168.0.1/24        md5

\end{verbatim}
\pagebreak
\section{Postgres Tunnel für pgadmin}
\begin{verbatim}
pgadmin wird verwendet um eine GUI Oberfläche für Postgresql Datenbanken zu 
haben. Da wegen der oben bereits erwähnten Firewall meist nur von lokal aus 
verbunden werden kann, benötigt es einen Tunnel um eine Verbindung herzu-
stellen.

der Tunnel wird wie gewöhnlich über SSH gestartet:

ssh -L <LokalerPort>:localhost:<5432(standard bei psql)> username@remote_ip 
\end{verbatim}

\section{Datenbank Passwort to md5}
\begin{verbatim}
UPDATE <TABLE> SET <ATTRIBUTE>=md5('pass') WHERE <ATTRIBUTE>='<VALUE>';
\end{verbatim}

\section{ZKS Karten und Benutzer Import}
\begin{verbatim}
ZKS Karten Importieren

Karten per insert Statement einfügen:

--insert into srv_user_cards (karten_nr, gesperrt) 
VALUES ('0000000000000000001', 'false');


Benutzer Importieren(letzte 3 Stellen der 
Karten Nummer = Benutzername(nachname))

-- SELECT * FROM srv_user;

-- INSERT INTO srv_user (name, vorname, firma)
-- SELECT substring(karten_nr FROM 17) as name, 
'Karte' as vorname, '' as firma FROM srv_user_cards;


Benutzer den Mandanten zuweisen

-- DELETE FROM srv_user2mandant;

--INSERT INTO srv_user2mandant (user_id, mandanten_id)
-- SELECT user_id, 2 AS mandanten_id FROM srv_user;
-- SELECT name FROM srv_user;



Karten Mandanten zuweisen

-- DELETE FROM srv_card2mandant;
INSERT INTO srv_card2mandant (card_id, mandanten_id)
SELECT card_id, 2 as mandanten_id FROM srv_user_cards;


Karten den Benutzern zuweisen. 
Karten bei denen die letzten 3 Stellen mit den Benutzernamen 
übereinstimmen(karte.001 = benutzer.001) werden miteinandern verknüpft.

UPDATE srv_user_cards uc
SET uc.user_id = (
SELECT u.user_id FROM srv_user u 
WHERE u.name=substring(uc.karten_nr FROM 17));


Gruppen zu Mandanten hinfügen

insert into srv_leser_gruppen2mandant (leser_group_id, mandanten_id)
select leser_group_id, 2 as mandanten_id from srv_leser_gruppen lg
EXCEPT 
select leser_group_id, mandanten_id from srv_leser_gruppen2mandant lg2m
where lg2m.mandanten_id = 2;



\end{verbatim}
\end{document}
