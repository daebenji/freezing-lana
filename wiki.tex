\documentclass[a4paper,10pt]{report}
%\documentclass[a4paper,10pt]{scrartcl}

\usepackage[utf8]{inputenc}
\title{Benjis Dokumentation}
\author{Benjamin Ludwig}
\date{}

\pdfinfo{%
  /Title    ()
  /Author   () 
  /Creator  ()
  /Producer ()	
  /Subject  ()
  /Keywords ()
}

\begin{document}
 
\maketitle
\tableofcontents
\chapter{Monitoring}
\section{Icinga Zeitprofile}


\begin{verbatim}

es wird eine Zeitperiode definiert, in der Alarmiert werden soll. 
Diese Periode ist dann mit 'check_period' auf den einzelnen Host 
oder Service anzuwenden. 
Im Beispiel soll immer alarmiert werden, AUSER von 05:00-06:25 
jeden Tag. 

Alarmierung für bestimmten Zeitpunkt abschalten:
\end{verbatim}

{\Large
\begin{verbatim}
define timeperiod {

     timeperiod_name 24x7_backup
     alias           immer-frueh
     sunday  00:00-05:00,06:25-24:00
     monday  00:00-05:00,06:25-24:00
     tuesday 00:00-05:00,06:25-24:00
     wednesday       00:00-05:00,06:25-24:00
     thursday        00:00-05:00,06:25-24:00
     friday  00:00-05:00,06:25-24:00
     saturday        00:00-05:00,06:25-24:00
}

\end{verbatim}
}
\chapter{sonstige Hacks}
\section{Unter Ubuntu jffs2-images mounten}
\begin{verbatim}
sudo apt-get install mtd-tools
sudo modprobe -v mtd
sudo modprobe -v jffs2
sudo modprobe -v mtdram total_size=256000 erase_size=256
sudo modprobe -v mtdchar
sudo modprobe -v mtdblock
sudo dd if=<deinImage.img> of=/dev/mtd0
sudo mount -t jffs2 /dev/mtdblock0 <deinPfadWoEsHinSoll>
\end{verbatim}

\section{Sed spielerei die Erste}
\begin{verbatim}
Achtung mit den Hochkommas!
Zeile an bestimmter Position einfügen(hier zeile 12) und dazu 
noch huebsch mit Tabulatoren formatieren:
sed ‘12i\\tTEXT\t\t\tMEHRTEXT’ <Datei>
\end{verbatim}
\section{Tunnel bauen}
\begin{verbatim}
#!/bin/bash
#build the tunnel to remote_ip via host
ssh -N -L <local_port>:<remote_ip>:<remote_port user@host &
#connect to host, via local port
ssh -p <local_port> <user>@localhost
#tunnel a remote port to another machine while using an existing tunnel
ssh -p <local_port> root@localhost -L localhost:8080:192.168.1.1:80

#scp durch bestehenden Tunnel
scp -P <local_port> <datei> root@localhost:<remote_pfad>
#oder vom remote host holen
scp -P <local_port> root@localhost:<remote_pfad> <lokaler_pfad>

\end{verbatim}

\section{expect-scripts}
\begin{verbatim}
#!/usr/bin/expect

if {$argc != 1} {
    send_user "\tusage: $argv0 <ip-address>\n"
    exit
}

set IPADDRESS [lindex $argv 0]

# security: write password to root only readable file in e.g. /root/authfiles
# so you may use this password here by:
#
#set PASSWORD_DIR   /root/authfiles
#set PASSWORD_FILE  "pwd-${IPADDRESS}"
#set status [catch { exec cat ${PASSWORD_DIR}${PASSWORD_FILE} } PASSWORD]
#
# alternatively set password simply here
set PASSWORD "<password>"

spawn /usr/bin/ssh admin@${IPADDRESS}

while (1) {
    expect {
        "password:" {			
            send "${PASSWORD}\n"
            break
        }
        # this is useful, if ssh connects first time to IPADDRESS
        "connecting (yes/no)?" { send "yes\n" }
    }
}
expect "ES-2024PWR#" { send "show hardware-monitor c\n" }
expect "ES-2024PWR#" { send "exit\n" }

\end{verbatim}

\section{rsync-magic}
\begin{verbatim}
logger -t Backup "begin incremental backup of <Directory>"
# incremental backup of /etc/apache2/*
rsync -chavz	P --stats /etc/apache2 \
<user>@<server>:<path_on_remote_host>
logger -t Backup "incremental backup done"
\end{verbatim}

\end{document}