\documentclass[a4paper,10pt]{report}
%\documentclass[a4paper,10pt]{scrartcl}

\usepackage[utf8]{inputenc}
\title{Benjis Dokumentation}
\author{Benjamin Ludwig}
\date{}

\pdfinfo{%
  /Title    ()
  /Author   () 
  /Creator  ()
  /Producer ()	
  /Subject  ()
  /Keywords ()
}

\begin{document}
 
\maketitle
\tableofcontents
\chapter{Monitoring}
\section{Icinga Zeitprofile}


\begin{verbatim}

es wird eine Zeitperiode definiert, in der Alarmiert werden soll. 
Diese Periode ist dann mit 'check_period' auf den einzelnen Host 
oder Service anzuwenden. 
Im Beispiel soll immer alarmiert werden, AUSER von 05:00-06:25 
jeden Tag. 

Alarmierung für bestimmten Zeitpunkt abschalten:
\end{verbatim}

{\Large
\begin{verbatim}
define timeperiod {

     timeperiod_name 24x7_backup
     alias           immer-frueh
     sunday  00:00-05:00,06:25-24:00
     monday  00:00-05:00,06:25-24:00
     tuesday 00:00-05:00,06:25-24:00
     wednesday       00:00-05:00,06:25-24:00
     thursday        00:00-05:00,06:25-24:00
     friday  00:00-05:00,06:25-24:00
     saturday        00:00-05:00,06:25-24:00
}

\end{verbatim}
}
\chapter{sonstige Hacks}
\section{Unter Ubuntu jffs2-images mounten}
\begin{verbatim}
sudo apt-get install mtd-tools
sudo modprobe -v mtd
sudo modprobe -v jffs2
sudo modprobe -v mtdram total_size=256000 erase_size=256
sudo modprobe -v mtdchar
sudo modprobe -v mtdblock
sudo dd if=<deinImage.img> of=/dev/mtd0
sudo mount -t jffs2 /dev/mtdblock0 <deinPfadWoEsHinSoll>
\end{verbatim}

\section{Sed spielerei die Erste}
\begin{verbatim}
Achtung mit den Hochkommas!
Zeile an bestimmter Position einfügen(hier zeile 12) und dazu 
noch huebsch mit Tabulatoren formatieren:
sed ‘12i\\tTEXT\t\t\tMEHRTEXT’ <Datei>
\end{verbatim}

\end{document}